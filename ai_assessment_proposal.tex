\documentclass[10pt,a4paper]{article}
\usepackage[utf8]{inputenc}
\usepackage{geometry}
\usepackage{amsmath}
\usepackage{amsfonts}
\usepackage{amssymb}
\usepackage{graphicx}
\usepackage{enumitem}
\usepackage{xcolor}
\usepackage{hyperref}
\usepackage{fancyhdr}
\usepackage{titlesec}

\geometry{margin=0.5in}
\pagestyle{fancy}
\fancyhf{}
\fancyhead[L]{\scriptsize AI-Based Assessment Analysis System}
\fancyhead[R]{\scriptsize Jahangirnagar University}
\fancyfoot[C]{\thepage}

\titleformat{\section}{\bfseries\color{blue}}{\thesection}{0.3em}{}
\titleformat{\subsection}{\bfseries\color{darkgray}}{\thesubsection}{0.3em}{}

\title{\textbf{AI-Based Assessment Analysis System}\\
       \smallskip
       Project Proposal for MSc Program}
\author{Department of Computer Science and Engineering\\
        Jahangirnagar University}
\date{}

\begin{document}

\maketitle
\vspace{-20pt}

\section*{Introduction}
\subsection*{Project Overview}
This project develops an AI-based assessment system for speaking, writing, and OCR evaluations. It provides comprehensive scoring (sophistication, vocabulary, content, grammar, overall, meaningfulness) with AI feedback. Features dual panels: instructors create assessments, students conduct them, both view visualization reports.

\section*{Objectives}
\begin{itemize}[leftmargin=*,noitemsep,topsep=0pt]
    \item Develop AI-powered assessment for speaking, writing, and OCR
    \item Implement multi-dimensional scoring algorithms
    \item Create dual-panel interface for instructors and students
    \item Provide detailed visualization reports
    \item Integrate AI-generated feedback
\end{itemize}

\section*{System Architecture}
Modular architecture with:
\begin{itemize}[leftmargin=*,noitemsep,topsep=0pt]
    \item Frontend: Web interface (React.js)
    \item Backend: Server (Node.js/Express)
    \item AI Services: OpenAI APIs
    \item Database: MongoDB
    \item Visualization: Chart.js
\end{itemize}

\section*{AI Assessment Methods}
\subsection*{Speaking Assessment}
Uses OpenAI's Whisper API for speech-to-text conversion. Evaluates pronunciation, fluency, vocabulary sophistication, grammar, and content relevance.

Process: Audio capture → Preprocessing → Speech recognition → Feature extraction → GPT-based evaluation → Scoring → Feedback.

\subsection*{Writing Assessment}
Analyzes grammar/syntax, vocabulary, content coherence, style/tone, and originality using GPT models.

Process: Text preprocessing → Feature extraction → Grammar checking → Semantic analysis → Multi-dimensional scoring → Feedback.

\subsection*{OCR Assessment}
Converts handwritten text using computer vision. Applies writing assessment algorithms post-conversion.

Process: Image preprocessing → Text detection → Character recognition → Layout analysis → Post-processing → Content evaluation.

\section*{Scoring Framework}
Multi-dimensional scoring:
\begin{itemize}[leftmargin=*,noitemsep,topsep=0pt]
    \item Sophistication: Language complexity
    \item Vocabulary: Word choice/diversity
    \item Content: Relevance/completeness
    \item Grammar: Syntactic correctness
    \item Overall: Composite weighted score
    \item Meaningfulness: Response value/insight
\end{itemize}

\section*{Implementation Plan}
14-week timeline:
\begin{enumerate}[leftmargin=*,noitemsep,topsep=0pt]
    \item Weeks 1-2: Requirements and design
    \item Weeks 3-6: Core system development
    \item Weeks 7-9: AI integration
    \item Weeks 10-12: Visualization/testing
    \item Weeks 13-14: Documentation/deployment
\end{enumerate}

\section*{Expected Outcomes}
Automated assessment reducing evaluation time, objective scoring, performance analytics, personalized feedback, and interactive visualization reports.

\section*{Conclusion}
This system demonstrates AI application in educational assessment, showcasing full-stack development and educational technology design skills for MSc program.

\end{document}